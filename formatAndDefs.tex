\documentclass[BCOR1cm, twoside, openright, titlepage, bibtotoc, abstracton]{scrartcl}

\usepackage[german,english]{babel}
\usepackage[T1]{fontenc}
\usepackage[utf8]{inputenc}
\usepackage{amsmath,amstext,amssymb}
\usepackage{geometry}
\usepackage{lmodern}
\usepackage{titlesec}
\usepackage[a4paper,pagebackref,hyperindex=true]{hyperref}
\usepackage{caption}
\usepackage{subfigure}
\usepackage{booktabs}
\usepackage{pdfpages}
\usepackage{appendix}
\usepackage{minitoc}
\usepackage{algorithm}

\usepackage{algorithmic}
\renewcommand{\algorithmicrequire}{\textbf{Input:}}
\renewcommand{\algorithmicensure}{\textbf{Output:}}

\usepackage[scaled=0.8]{beramono}

\usepackage{fancyhdr}                   
\setlength{\headheight}{25.2232pt}

\selectlanguage{english}

% listings
\usepackage{listings}
\lstset{ %
  language=C++, % choose the language of the code
  basicstyle=\small\ttfamily, % the size of the fonts that are used for the code
  numbers=left, % where to put the line-numbers
  numberstyle=\small\ttfamily\color[rgb]{0.6,0.6,0.6}, % the size of the fonts that are used for the line-numbers
  stepnumber=1, % the step between two line-numbers. If it's 1 each line
  xleftmargin=4mm,
  % will be numbered
  numbersep=5pt, % how far the line-numbers are from the code
  backgroundcolor=\color{white}, % choose the background color. You must add \usepackage{color}
  showspaces=false, % show spaces adding particular underscores
  showstringspaces=false, % underline spaces within strings
  showtabs=false, % show tabs within strings adding particular underscores
  %frame=l, % adds a frame around the code
  frame=single,
  tabsize=4, % sets default tabsize to 2 spaces
  breaklines=true, % sets automatic line breaking
  breakatwhitespace=false, % sets if automatic breaks should only happen at whitespace
  % also try caption instead of title
  escapeinside={@}{@}, % if you want to add a comment within your code
  morekeywords={*,...,omp, parallel,__local,__global,__kernel,value_type}, % if you want to add more keywords to the set
  keywordstyle=\color[rgb]{0,0,1},
  commentstyle=\color[rgb]{0.133,0.545,0.133}\textit,
  stringstyle=\color[rgb]{0.627,0.126,0.941},
}
\newcommand{\matlab}[2] {\vspace{3mm}\lstinputlisting[title={#2}]{#1}}

% Links in pdf
\usepackage{color}
\newcommand{\blue}{ \color{blue} }
\definecolor{linkcol}{rgb}{0,0,0.4}
\definecolor{citecol}{rgb}{0.5,0,0}

\hypersetup
{
bookmarksopen=true,
pdftitle="Master Thesis",
pdfauthor="Max Muster",
pdfsubject="Title of the Document", %subject of the document
%pdftoolbar=false, % toolbar hidden
pdfmenubar=true, %menubar shown
pdfhighlight=/O, %effect of clicking on a link
colorlinks=true, %couleurs sur les liens hypertextes
pdfpagemode=None, %aucun mode de page
pdfpagelayout=SinglePage, %ouverture en simple page
pdffitwindow=true, %pages ouvertes entierement dans toute la fenetre
linkcolor=linkcol, %couleur des liens hypertextes internes
citecolor=citecol, %couleur des liens pour les citations
urlcolor=linkcol %couleur des liens pour les url
}

% nicer backref links
\renewcommand*{\backref}[1]{}
\renewcommand*{\backrefalt}[4]{%
\ifcase #1 %
(Not cited.)%
\or
(Cited on page~#2.)%
\else
(Cited on pages~#2.)%
\fi}
\renewcommand*{\backrefsep}{, }
\renewcommand*{\backreftwosep}{ and~}
\renewcommand*{\backreflastsep}{ and~}


\newcommand{\ve}[1]{
	\mathbf{#1}
	%\vec{#1}
}

\newcommand{\grad}{ \textbf{\,grad\,} }
\providecommand{\argmin}{\operatorname*{argmin}} % operatorname makes _{..} appear centered
\newcommand{\dx}{ \,\ve{dx} }

\renewcommand{\a}{\ve{a}}
\renewcommand{\b}{\ve{b}}
\renewcommand{\c}{\ve{c}}
\renewcommand{\d}{\ve{d}}
\newcommand{\e}{\ve{e}}
\newcommand{\f}{\ve{f}}
\newcommand{\g}{\ve{g}}
\newcommand{\h}{\ve{h}}
\renewcommand{\i}{\ve{i}}
\renewcommand{\j}{\ve{j}}
\renewcommand{\k}{\ve{k}}
\renewcommand{\l}{\ve{l}}
\newcommand{\m}{\ve{m}}
\newcommand{\n}{\ve{n}}
\renewcommand{\o}{\ve{o}}
\newcommand{\p}{\ve{p}}
\newcommand{\q}{\ve{q}}
\renewcommand{\r}{\ve{r}}
\newcommand{\s}{\ve{s}}
\renewcommand{\t}{\ve{t}}
\renewcommand{\u}{\ve{u}}
\renewcommand{\v}{\ve{v}}
\newcommand{\w}{\ve{w}}
\newcommand{\x}{\ve{x}}
\newcommand{\y}{\ve{y}}
\newcommand{\z}{\ve{z}}

\newcommand{\A}{\ve{A}}
\newcommand{\B}{\ve{B}}
\newcommand{\C}{\ve{C}}
\newcommand{\D}{\ve{D}}
\newcommand{\E}{\ve{E}}
\newcommand{\F}{\ve{F}}
\newcommand{\G}{\ve{G}}
\renewcommand{\H}{\ve{H}}
\newcommand{\I}{\ve{I}}
\newcommand{\J}{\ve{J}}
\newcommand{\K}{\ve{K}}
\renewcommand{\L}{\ve{L}}
\newcommand{\M}{\ve{M}}
\newcommand{\N}{\ve{N}}
\renewcommand{\O}{\ve{O}}
\renewcommand{\P}{\ve{P}}
\newcommand{\Q}{\ve{Q}}
\newcommand{\R}{\ve{R}}
\renewcommand{\S}{\ve{S}}
\newcommand{\T}{\ve{T}}
\newcommand{\U}{\ve{U}}
\newcommand{\V}{\ve{V}}
\newcommand{\W}{\ve{W}}
\newcommand{\X}{\ve{X}}
\newcommand{\Y}{\ve{Y}}
\newcommand{\Z}{\ve{Z}}

\hyphenation{inter-grid}

\newenvironment{packed_item}{
\vspace{-2mm}
\begin{itemize}
  \setlength{\itemsep}{1pt}
  \setlength{\parskip}{0pt}
  \setlength{\parsep}{0pt}
}{\vspace{-2mm}\end{itemize}}

% paragraph line skip
%\setlength{\parskip}{\baselineskip}
\setlength{\parindent}{0pt}
\setlength{\parskip}{2ex}


% Macro for 'List of Symbols', 'List of Notations' etc...
\def\listofsymbols{%%%%%%%%%%%%%%%%%%%%%%%
%Sample List of Symbols
%%%%%%%%%%%%%%%%%%%%%%%
\begin{tabbing}
% YOU NEED TO ADD THE FIRST ONE MANUALLY TO ADJUST THE TABBING AND SPACES
\parbox{20mm}{$\I$}\=\parbox{115mm}{General Image\dotfill \pageref{symbol:I}}\\
%ADD THE REST OF SYMBOLS WITH THE HELP OF MACRO
\addsymbol I_{i,j}:			{$i$-th Pixel and $j$-th Pixel an Image $\I$}{symbol:Iij}
\addsymbol I(x,y,t):		{Intensity of an image $\I$ at position $(x,y)$ at time $t$, used for anisotropic filtering}{symbol:Ixyt}
\addsymbol \A:				{Binary Image}{symbol:BinI}
\addsymbol \B:				{Binary structuring element}{symbol:struct}
\addsymbol \F:				{Generic Feature}{symbol:F}
\addsymbol \F(I_{i,j}):		{Result of a generic feature applied on a pixel}{symbol:FIij}
\addsymbol \G:				{Gray Level Co-Occurrence Matrix}{symbol:G}
\addsymbol g_{ij}:			{$(i,j)$-Element of $\G$}{symbol:gij}
\addsymbol T:				{General threshold used for a segmenting a gray-scale image}{symbol:thres}
\addsymbol T_i:				{$i$-th threshold used for a segmenting a gray-scale image with more than two pixel-classes, ($i \leq j \Rightarrow T_i \leq T_j$)}{symbol:thres_i}

\addsymbol G:			{Graph}{symbol:graph}
\addsymbol V:			{Set of vertices belonging to a graph $G$}{symbol:vertices}
\addsymbol E:			{Set of edges belonging to a graph $G$}{symbol:edges}
% .
% .
% .
% ALWAYS KEEP THE FOLLOWING LINE
\end{tabbing}

 \clearpage}
\def\addsymbol #1: #2#3{$#1$ \> \parbox{115mm}{#2 \dotfill \pageref{#3}}\\}
\def\newnot#1{\label{#1}} 


\newcommand{\subfigureautorefname}{\figureautorefname}


\def\sectionautorefname{Chapter}
\def\subsectionautorefname{Section}
\def\subsubsectionautorefname{Section}
\def\algorithmautorefname{Algorithm}

\usepackage{hyperref}

\font\capfonta=cmbx12 at 32 pt % or yinit, or...?
\newbox\capbox \newcount\capl \def\a{A}
\def\docappar{\medbreak\noindent\setbox\capbox\hbox{\capfonta\a\hskip0.10em}%
\hangindent=\wd\capbox%
\capl=\ht\capbox\divide\capl by\baselineskip\advance\capl by1\hangafter=-\capl%
\hbox{\vbox to8pt{\hbox to0pt{\hss\box\capbox}\vss}}}
\def\cappar{\afterassignment\docappar\noexpand\let\a }

\geometry{a4paper, top=45mm, left=27mm, right=27mm, bottom=50mm}
\oddsidemargin 15mm
\evensidemargin 5mm
%\textwidth 135mm
\setlength{\textwidth}{140mm}

%% Packages für Grafiken & Abbildungen %%%%%%%%%%%%%%%%%%%%%%
\usepackage{graphicx} %%Zum Laden von Grafiken
\graphicspath{{.}{images/}}
%\usepackage{tikz} %%Vektorgrafiken aus LaTeX heraus erstellen




